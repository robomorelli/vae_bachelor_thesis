\begin{abstract}
 Questa tesi ha l'obiettivo di verificare l'utilità di una particolare tipologia di algoritmo di Machine Learning, i $\textit{Variational Autoencoders}$, nel campo della fisica delle alte energie per la ricerca oltre il Modello Standard (BSM). Nello specifico viene addestrato il VAE su dei segnali di background coerenti con il Modello Standard, in modo che sia in grado di rilevare eventuali segnali BSM come delle anomalie.\\
 Il lavoro è organizzato in tre parti: nella prima parte viene spiegato il motivo per cui si impiega il ML per la discriminazione degli eventi di segnale da quelli di background, nella seconda si presentano le metodologie più comuni, compresi i Variational Autoencoders, e le caratteristiche fondamentali degli algoritmi ML, mentre nella terza parte vengono presentati i risultati del processo di apprendimento del VAE per la ricerca BSM, in particolare per la ricerca della coppia di particelle Chargino-Gluino (previste dalla teoria SUSY).
\end{abstract}