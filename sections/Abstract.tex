\begin{abstract}
	
Negli ultimi anni sono stati sviluppati sistemi sempre più complessi di analisi di grandi quantità di dati: nel campo della fisica delle alte energie, il grande numero di eventi prodotti in collisionatori rendono questi sistemi molto utili per la ricerca di eventi rari. In questa tesi da prima verrà descritta l’evoluzione degli algoritmi utilizzati per l’analisi multivariata di campioni molto estesi di dati; ci si focalizzerà in particolare su sistemi di machine learning, con particolare attenzione sui Variational Autoencoders e la loro applicazione nel campo della fisica delle alte energie. Verranno quindi presentati i risultati dell'applicazione di un Variational Autoencoder per la ricerca di fisica oltre il Modello Standard. Verrà descritto il processo di addestramento di tale algoritmo, effettuato su campioni di eventi simulati secondo le predizioni del Modello Standard, e verrà valutata la sensibilità a processi di produzione elettrodebole di particelle supersimmetriche (qui va inseritoil processo), in cui <qui va descritta la catena di decadimento>. Con l’applicazione dell’algoritmo descritto sono stati ottenuti limiti <inserire qui il risultato sulla sensibilita’>.

\end{abstract}