\begin{abstract}
	
Negli ultimi anni sono stati sviluppati sistemi sempre più complessi di analisi di grandi quantità di dati: nel campo della fisica delle alte energie, il grande numero di eventi prodotti in collisionatori rendono questi sistemi molto utili per la ricerca di eventi rari. In questa tesi da prima verrà descritta l’evoluzione degli algoritmi utilizzati per l’analisi multivariata di campioni molto estesi di dati; ci si focalizzerà in particolare su sistemi di machine learning, con particolare attenzione sui Variational Autoencoders e la loro applicazione nel campo della fisica delle alte energie. Verranno quindi presentati i risultati dell'applicazione di un Variational Autoencoder per la ricerca di fisica oltre il Modello Standard (BSM). Verrà descritto il processo di addestramento di tale algoritmo, effettuato su campioni di eventi simulati secondo le predizioni del Modello Standard, e verrà valutata la sensibilità a processi di produzione elettrodebole di particelle supersimmetriche ($ pp \rightarrow \tilde{\chi}_1^\pm + \tilde{\chi}_2^0, \tilde{\chi}_1^\pm \rightarrow W + \tilde{\chi}_1^0, \tilde{\chi}_2^0 \rightarrow h + \tilde{\chi}_1^0$), dove dalla collisione di due protoni si ottengono un chargino $\tilde{\chi}_1^\pm$ ed un neutralino pesante $\tilde{\chi}_2^0$; in seguito il chargino decade in un bosone W ed un neutralino leggero  $\tilde{\chi}_1^0$, mentre il neutralino pesante decade in un Bosone di Higgs h ed in un altro neutralino leggero. La ricerca è volta all'osservazione dell'elettrone e del muone prodotti dal decadimento del W ed alla coppia di b-quarks, prodotti dal decadimento dell' h. \\
Con l'applicazione dell'algoritmo descritto è stato ottenuto il limite di 800 GeV per la massa del chargino per la sensibilità al segnale BSM, nell'ipotesi di massa nulla del neutralino leggero.


\end{abstract}