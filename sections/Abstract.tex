\begin{abstract}
	
Negli ultimi anni sono stati sviluppati sistemi sempre più complessi di analisi di grandi quantità di dati: nel campo della fisica delle alte energie, il grande numero di eventi prodotti in collisionatori rendono questi sistemi molto utili per la ricerca di eventi rari. In questa tesi da prima verrà descritta l’evoluzione degli algoritmi utilizzati per l’analisi multivariata di campioni molto estesi di dati; ci si focalizzerà in particolare su sistemi di machine learning, con particolare attenzione sui Variational Autoencoders e la loro applicazione nel campo della fisica delle alte energie. Verranno quindi presentati i risultati dell'applicazione di un Variational Autoencoder per la ricerca di fisica oltre il Modello Standard. Verrà descritto il processo di addestramento di tale algoritmo, effettuato su campioni di eventi simulati secondo le predizioni del Modello Standard, e verrà valutata la sensibilità a processi di produzione elettrodebole di particelle supersimmetriche ($\tilde{\chi}_1^\pm + \tilde{\chi}_2^0 \rightarrow W + h + 2\tilde{\chi}_1^0$), in cui un chargino $\tilde{\chi}_1^\pm$ decade in un bosone W ed in un neutralino leggero $\tilde{\chi}_1^0$, mentre un neutralino pesante $\tilde{\chi}_2^0$ decade in un secondo neutralino leggero ed in un bosone di Higgs h. Con l'applicazione dell'algoritmo descritto sono stati ottenuti i limiti...

\color{red}
che risultati devo mettere (risultato sulla sensibilità) ?
\color{black}

\end{abstract}