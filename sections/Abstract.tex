\begin{abstract}
 Questa tesi ha l'obiettivo di verificare l'utilità di una particolare tipologia di algoritmo di Machine Learning, i $\textit{Variational Autoencoders}$, nel campo della fisica delle alte energie per la ricerca oltre il Modello Standard (BSM). Nello specifico viene addestrato il VAE su dei segnali di background coerenti con il Modello Standard, in modo che sia in grado di rilevare eventuali segnali BSM come delle anomalie.\\
 Il lavoro è organizzato in tre parti: nella prima parte viene spiegato il motivo per cui si impiega il ML per la discriminazione degli eventi di segnale da quelli di background, nella seconda si presentano le metodologie più comuni, compresi i Variational Autoencoders, e le caratteristiche fondamentali degli algoritmi ML, mentre nella terza parte vengono presentati i risultati del processo di apprendimento del VAE per la ricerca BSM, in particolare per la ricerca della coppia di particelle Chargino-Gluino (previste dalla teoria SUSY).

Generale: non stiamo provando a mostrare l’utilita’ di uno strumento (e’ stato gia’ mostrato in articoli pubblicati), quanto valutarne l’impatto su un caso particolare. Inoltre dato che una parte consistente della tesi e’ di tipo descrittivo e compilativo io scriverei qualcosa su queste linee:

Negli ultimi anni sono stati sviluppati sistemi sempre più complessi di analisi di grandi quantità di dati: nel campo della fisica delle alte energie, il grande numero di eventi prodotti in collisionatori rendono questi algoritmi molto utili per la ricerca di eventi rari. In questa tesi da prima verrà descritta l’evoluzione degli algoritmi utilizzati per l’analisi multivariata di campioni molto estesi di dati. Ci focalizzeremo in particolare su sistemi di machine learning, con particolare attenzione sui Variational Autoencoder e la loro applicazione nel campo della fisica delle alte Energie. Verranno quindi presentati i risultati della applicazione di un Variational Auto encoder per la ricerca di fisica oltre il modello standard.  Descriveremo il training di tale algoritmo, effettuato su campioni di eventi simulati secondo le predizioni del modello standard, e verra’ valutata la sensibilita’ a processi di produzione elettrodebole di particelle supersimmetriche (qui va inseritoil processo), in cui <qui va descritta la catena di decadimento>. Con l’applicazione dell’algoritmo descritto sono stati ottenuti limiti <inserire qui il risultato sulla sensibilita’>

 
 
\end{abstract}