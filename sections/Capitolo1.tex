\section{Analisi multivariata e Machine Learning}
\label{analisi multivariata e ML}

	Quando si parla di analisi dati ci si può essenzialmente ricondurre a tre macro-categorie di operazioni:

\begin{enumerate}
	\item CLASSIFICAZIONE \\
	Questa tipologia è probabilmente la principale quando si ha a che fare con la fisica delle alte energie e consiste nell'associare un evento/oggetto  ad una categoria. Nel caso specifico di questa trattazione l'obiettivo sarà esattamente quello di classificare gli eventi/oggetti nelle due categorie di background e segnale.

	\item STIMA DI PARAMETRI \\
	In questa tipologia ricadono tutti quei processi attraverso i quali si estraggono dei parametri (ad esempio la massa di una tipologia di particelle) attraverso un fitting del modello teorico con i dati sperimentali;
	
	\item STIMA DI FUNZIONI \\
	Si ricava una funzione continua di una o più variabili a partire dai dati sperimentali.
\end{enumerate}

Questa trattazione si concentrerà essenzialmente sulle possibili metodologie di classificazione, proprio perché l'obiettivo è il discernimento fra segnale e background alla ricerca di nuova fisica. \\
Nelle prime righe di questo capitolo è stato introdotto il termine $\textit{evento}$ o $\textit{oggetto}$, senza meglio specificare come questo fosse collegato ai dati. Un evento può essere pensato come una collezione di dati e quindi lo si può rappresentare come un vettore in uno spazio n-dimensionale: 
\begin{equation}
\textbf{x} = (x_{1},...,x_{n})
\end{equation}
In realtà l'utilizzo del termine vettore è improprio ogni qual volta si abbia a che fare con componenti (i dati) disomogenee tra loro, tuttavia lo si continuerà ad utilizzare per una questione di comodità tenendo a mente questa specifica. Nelle pagine successive con i termini evento, oggetto, vettore di input e pattern ci si riferirà sempre alla stessa entità appena introdotta. \\