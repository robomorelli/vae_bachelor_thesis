
\section{Conclusioni}
\label{sec:conclusioni}

In questa tesi è stato presentato un possibile approccio alla ricerca BSM, ossia alla ricerca di nuova fisica oltre il Modello Standard. Sono state presentate le varie possibilità che si hanno a disposizione per la discriminazione degli eventi di segnale, cioè quelli riconducibili a nuova fisica, e gli eventi di fondo, cioè quelli riconducibili al Modello Standard già noto. Il più avanzato di questi approcci prevede l'utilizzo di algoritmi di apprendimento automatico (\textit{machine learning}), dei quali è stata fatta un'ampia panoramica delle caratteristiche e delle metodologie più note ed utilizzate. In particolare ci si è focalizzati su un metodo di apprendimento non supervisionato, il \textit{Variational Autoencoder} (VAE), per verificare se possa essere utilizzato nel processo di discriminazione fra segnale e fondo; per fare ciò il VAE è stato addestrato sugli eventi generati attraverso una simulazione Montecarlo ed i risultati finali sono stati assolutamente positivi. Nello specifico si è osservato che, a seguito del processo di addestramento sui dati di fondo, il VAE può essere utilizzato per la ricerca di eventi di segnale grazie al fatto che tali eventi, una volta ricostruiti, hanno un errore di ricostruzione tendenzialmente maggiore rispetto agli eventi di fondo. Di quest'ultimo aspetto è stata compiuta una dimostrazione qualitativa utilizzando la distribuzione della \textit{loss} ed una quantitativa, andando a verificare per quali combinazioni delle masse delle due particelle ricercate (chargino e neutralino) il VAE fosse in grado di attuare la discriminazione.\\
In ultima analisi è stato effettuato un tentativo di ottimizzazione del processo tramite una variazione degli iperparametri, ovvero pesando in maniera diversa le variabili che compongono i pattern ed il risultato è stato incoraggiante perché è emerso che tale variazione degli iperparametri permette di rendere il VAE discriminante per delle combinazioni di masse per le quali precedentemente non lo era.\\
Quindi, per ciò che è stato appena detto, è stata dimostrata l'utilità del VAE per una ricerca di processi supersimmetrici sulla SUSY. In conclusione, come possibili sviluppi futuri, si potrebbe verificare se il VAE così addestrato risulti discriminante anche per eventi di segnale riconducibili a teorie diverse dalla SUSY.