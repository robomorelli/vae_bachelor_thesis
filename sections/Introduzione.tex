%
\section{Introduzione}
\label{sec:introduzione}
%
Il modello Standard (SM) è una teoria che ha avuto grande successo, ed è riuscita a spiegare e prevedere molti dei fenomeni osservati a livello subnucleare. Sebbene la scoperta del \textit{Bosone di Higgs} \cite{Bosone_di_Higgs} abbia confermato la rottura di simmetria elettrodebole, ha portato in primo piano lo \textit{hierarchy problem} \cite{PG1,PG2,PG3,PG4}, cioè la grande differenza tra l'accoppiamento elettrodebole e quello gravitazionale. Inoltre, nonostante l'evidenza di materia oscura nell'universo, lo SM non prevede alcuna particella che possa giustificare la presenza della materia oscura osservata, ad esempio negli aloni galattici. \\
Un'ipotesi è che il MS rappresenti il limite a basse energie di una teoria più complessa, quindi il \textit{Large Hadron Collider} rappresenta un ottimo strumento per la ricerca di nuove particelle molto massive, non previste dal Modello Standard.\\
Per esempio, la teoria della \textit{Supersimmetria} (SUSY), che è una estensione del MS, risolve il Problema della Gerarchia introducendo un nuovo fermione/bosone per ogni fermione/bosone del MS; inoltre tali particelle sarebbero stabili e poco interagenti e quindi costituirebbero delle ottime candidate per la spiegazione della materia oscura. \\
Per ottenere una buona reiezione del fondo accompagnata da una buona efficienza sul segnale si possono utilizzare  sistemi di analisi statistica molto complessi. Il machine learning (ML) rappresenta una serie di metodologie di natura statistica-computazionale che permettono di estrarre informazioni da enormi moli di dati senza la supervisione dell'analista. In fisica delle alte energie gli algoritmi di ML, attraverso l'apprendimento delle correlazioni tra le proprietà cinematiche delle particelle presenti in un evento, consentono di catalogare ciascun evento come affine al segnale o al fondo. In particolare i Variational Autoencoders (VAEs) \cite{Understanding_VAEs}, una cui applicazione è descritta nel capitolo \ref{fisica_BSM_VAEs}, si basano sulla riduzione della dimensione delle variabili che descrivono gli eventi, seguita da una fase di ricostruzione del campione. Nello specifico il VAE comprime ogni singolo evento di input che gli viene presentato non come un punto in uno spazio di dimensione minore (detto spazio latente), bensì come una distribuzione; si campiona quindi un punto nello spazio latente a partire da tale distribuzione, che viene ricostruito a seguito di un processo di decompressione. \\
Una volta addestrato il VAE a riconoscere e riprodurre le distribuzioni delle variabili cinematiche che descrivono gli eventi SM è possibile, confrontando questi risultati con i dati ottenuti dall'esperimento, osservare (o valutare il limite di esclusione) anomalie che possono essere dovute alla presenza di eventi non descritti dal Modello Standard.\\
Questa tesi è strutturata in tre capitoli: nel primo vengono spiegate le differenze fra gli algoritmi di machine learning ed i metodi di analisi multivariata e non, nel secondo vengono approfondite le metodologie di machine learning e si presentano i Variational Autoencoders, mentre nell'ultimo viene applicato quest'ultimo metodo nel campo della fisica delle alte energie.


\newpage






	


