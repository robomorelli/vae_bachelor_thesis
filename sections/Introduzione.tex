%
\section{Introduzione}
\label{sec:introduzione}
%
Il modello Standard (SM) e' una teoria che ha avuto grande successo, ed e' riuscita a spiegare e prevedere molti dei fenomeni osservati a livello subnucleare. D'altra parte, sebbene la scoperta del \textit{Bosone di Higgs} \cite{Bosone_di_Higgs} abbia confermato la rottura di simmetria elettrodebole [referenza a glashow salam weinberg], ha portato in primo piano lo \textit{hierarchy problem} \cite{PG1,PG2,PG3,PG4}, cioè la grande differenza tra l'accoppiamento elettrodebole e quello gravitazionale. Inoltre, nonostante l'evidenza di materia oscura nell'universo, lo SM non prevede alcuna particella che possa giustificare la presenza della materia oscura osservata, ad esempio negli aloni galattici. \\
Da queste considerazioni sembrerebbe essere ormai arrivati ad un punto di stallo per quanto riguarda il MS, tuttavia è evidente da ciò che è stato accennato precedentemente che rimangono aperte molte domande; una ipotesi è che il MS rappresenti il limite a basse energie di una teoria più complessa, quindi una serie di fenomeni o non avvengono alle attuali energie raggiungibili al \textit{Large Hadron Collider} oppure sono estremamente rari. \\
Per esempio, la teoria della \textit{Supersimmetria} (SUSY), che è una estensione del MS, risolve il Problema della Gerarchia introducendo un nuovo fermione/bosone per ogni fermione/bosone del MS; inoltre tali particelle sarebbero stabili e poco interagenti e quindi costituirebbero delle ottime candidate per la spiegazione della materia oscura. \\
Per ottenere una buona reiezione del fondo accompagnata da una buona efficienza sul segnale si possono utilizzare  sistemi di analisi statistica molto complessi. Il machine learning (ML) rappresenta una serie di metodologie di natura statistica-computazionale che permettono di estrarre informazioni da enormi moli di dati senza la supervisione dell'analista. In fisica delle alte energie gli algoritmi di ML, attraverso l'apprendimento delle correlazioni tra le proprietà cinematiche delle particelle presenti in un evento, consentono di catalogare ciascun evento come affine al segnale o al fondo. In particolare i Variational Autoencoders (VAEs) \cite{Understanding_VAEs}, una cui applicazione e' descritta nel capitolo \ref{fisica_BSM_VAEs}, si basano sulla riduzione della dimensione delle variabili che descrivono gli eventi, seguita da una fase di ricostruzione del campione. Nello specifico il VAE comprime ogni singolo evento di input che gli viene presentato non come un punto in uno spazio di dimensione minore (detto spazio latente), bensì come una distribuzione; si campiona quindi un punto nello spazio latente a partire da tale distribuzione, che viene ricostruito a seguito di un processo di decompressione. \\
Una volta addestrato il VAE a riconoscere e riprodurre le distribuzioni delle variabili cinematiche che descrivono gli eventi SM e' possibile, confrontando questi risultati con i dati ottenuti dall'esperimento, osservare (o valutare il limite di esclusione) anomalie che possono essere dovute alla presenza di eventi non descritti dal Modello Standard.


\color{red}
referenza per la rottura della simmetria (a glashow salam weinberg) \\
per ottenere una buona reiezione...\\

\color{black}

\newpage

%Fin dalla prima metà degli anni '60 era chiaro ai fisici che l'idea di una materia formato esclusivamente da elettroni, protoni e neutroni era limitante e non in grado di spiegare la moltitudine di particelle che erano ormai già state osservate. Per questo motivo nel 1964 Gell-Mann e Zweig proposero la Teoria dei Quark, che è stata arricchita negli anni successivi ed è oggi nota come teoria del Modello Standard. 
%Il Modello Standard [...] è la teoria che ha permesso l'unificazione di tre delle quattro interazioni fondamentali (forte, debole ed elettromagnetica) e, presumibilmente, ha raggiunto il suo massimo con la scoperta del \textit{Bosone di Higgs} \cite{Bosone_di_Higgs} nel 2012; tuttavia, mancando una formulazione dell'interazione gravitazionale, non può essere considerata una teoria del tutto. \\
%Ci sono poi alcuni problemi che non possono essere spiegati con il MS, come lo \textit{Hierarchy Problem} \cite{ProblemaGerarchia}
%o la presenza della $\textit{Black Mattrn}$ (BM, materia oscura) [...]. Per quanto riguarda la materia oscura risulta che nessuna delle particelle fondamentali del MS è una buona candidata a farne parte, per esempio se la BM fosse costituita da particelle cariche si sarebbe dovuta rilevare una qualche radiazione elettromagnetica proveniente dalle zone di universo nelle quali si stima esserci BM, ma così non è stato.  \\
%Da queste considerazioni sembrerebbe essere ormai arrivati ad un punto di stallo per quanto riguarda il MS, tuttavia è evidente da ciò che è stato accennato precedentemente che rimangono aperte molte domande; una ipotesi è che il MS rappresenti il limite a basse energie di una teoria più complessa, quindi una serie di fenomeni o non avvengono alle attuali energie raggiungibili al \textit{Large Hadron Collider} oppure sono estremamente rari. \\
%Per esempio, la teoria della \textit{Supersimmetria} (SUSY) [...], che è una estensione del MS, risolve il Problema della Gerarchia introducendo un nuovo fermione/bosone per ogni fermione/bosone del MS; inoltre tali particelle sarebbero stabili e poco interagenti e quindi costituirebbero delle ottime candidate per la spiegazione della materia oscura. \\
%Tutte queste considerazioni inducono a pensare che esista una fisica \textit{beyond the Standard Model} (BSM) e la grande sfida dei prossimi anni è quella di capire in che modo si possa indagarla. \\
%Il run3 del Large Hadron Collider è previsto per Maggio 2021, tuttavia non vi è stato un miglioramento notevole da un punto di vista energetico. Allo stesso tempo si stima che la produzione di dati sarà fino a dieci volte maggiore rispetto al run precedente, quindi la domanda è se sia possibile trattare in maniera innovativa questa enorme mole di dati per cercare di estrarre segnale di nuova fisica. \\
%Nello specifico la domanda è se sia possibile utilizzare delle metodologie di Machine Learning per la separazione del segnale dal background in modo da riuscire ad osservare eventuali segnali rari.\\
%Con il termine machine learning si intende una serie di metodologie di natura statistico-computazionale che permettono di estrarre informazione utile da enormi moli di dati, altrimenti difficilmente processabili dall'uomo.
%I dati, per la loro stessa natura, sono disomogenei e caotici, quindi risulta particolarmente complesso analizzarli per ottenerne dei risultati. Qui entra in gioco il machine learning, ovvero l'apprendimento automatico della "macchina", perché permette di trovare relazioni nascoste fra i dati autonomamente, ovvero senza la continua supervisione dell'essere umano. \\
%In particolare verrà affrontato un metodo di ML, il \textit{Variational Autoencoders} (VAEs), che si basa essenzialmente su un processo di diminuzione della dimensionalità dei dati ed una successiva fase di ricostruzione; tale algoritmo viene addestrato sui dati di background in modo che sia capace di riconoscere eventuali segnali di nuova fisica come delle anomalie.\\




	


