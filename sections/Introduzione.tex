%
\section{Introduzione}
\label{sec:introduzione}
%
Fin dalla prima metà degli anni '60 era chiaro ai fisici che l'idea di una materia formato esclusivamente da elettroni, protoni e neutroni era limitante e non in grado di spiegare la moltitudine di particelle che erano ormai già state osservate. Per questo motivo nel 1964 Gell-Mann e Zweig proposero la teoria dei quark e tale teoria è stata mano mano arricchita negli anni successivi ed è oggi nota come teoria del Modello Standard. 
Il Modello Standard [...] è la teoria che ha permesso l'unificazione di tre delle quattro interazioni fondamentali (forte, debole ed elettromagnetica) e, presumibilmente, ha raggiunto il suo massimo con la scoperta del $\textit{Bosone di Higgs}$[...] nel 2012; tuttavia, mancando una formulazione dell'interazione gravitazionale, non può essere considerata una teoria del tutto. \\
Ci sono poi alcuni problemi che non possono essere spiegati con il MS, come lo $\textit{hierarchy problem}$ [...]
o la presenza della $\textit{Black Mattrn}$ (BM, materia oscura) [...]. Per quanto riguarda la materia oscura risulta che nessuna delle particelle fondamentali del MS è una buona candidata a farne parte, per esempio se la BM fosse costituita da particelle cariche si sarebbe dovuta rilevare una qualche radiazione elettromagnetica proveniente dalle zone di universo nelle quali si stima esserci BM.  \\
Da queste considerazioni sembrerebbe essere ormai arrivati ad un punto di stallo per quanto riguarda il MS, tuttavia è evidente da ciò che è stato accennato precedentemente che rimangono aperte molte domande; una ipotesi è che il MS rappresenti il limite a basse energie di una teoria più complessa, quindi una serie di fenomeni o non avvengono alle attuali energie raggiungibili al $\textit{Large Hidron Collider}$ oppure sono estremamente rari. \\
Per esempio, la teoria della $\textit{Supersimmetria}$ (SUSY) [...], che è una estensione del MS, risolve il Problema della Gerarchia introducendo un nuovo fermione/bosone per ogni fermione/bosone del MS; inoltre tali particelle sarebbero stabili e poco interagenti e quindi costituirebbero delle ottime candidate per la spiegazione della materia oscura. \\
Tutte queste considerazioni inducono a pensare che esista una fisica $\textit{"behind the Standard Model" }$ (BSM) e la grande sfida dei prossimi anni è quella di capire in che modo andare ad indagarla. \\
Il run3 del Large Hadron Collider è previsto per Maggio 2021, tuttavia non vi è stato un miglioramento notevole da un punto di vista energetico. Allo stesso tempo si stima che la produzione di dati sarà fino a dieci volte maggiore rispetto al run precedente, quindi la domanda è se sia possibile trattare in maniera innovativa questa enorme mole di dati per cercare di estrarre segnale di nuova fisica. \\
Nello specifico la domanda è se sia possibile utilizzare delle metodologie di Machine Learning per la separazione del segnale dal background in modo da riuscire ad osservare eventuali segnali rari.\\
Con il termine machine learning si intende una serie di metodologie di natura statistico-computazionale che permettono di estrarre informazione utile da enormi moli di dati, altrimenti difficilmente processabili dall'uomo.
I dati, per la loro stessa natura, sono disomogenei e caotici, quindi risulta particolarmente complesso analizzarli per ottenerne dei risultati. Qui entra in gioco il machine learning, ovvero l'apprendimento automatico della "macchina", perché permette di trovare relazioni nascoste fra i dati autonomamente, ovvero senza la continua supervisione dell'essere umano. \\
In particolare verrà affrontato un metodo di ML, il $\textit{Variational Autoencoders}$ (VAEs), che si basa essenzialmente su un processo di diminuzione della dimensionalità dei dati ed una successiva fase di ricostruzione; tale algoritmo viene allenato sui dati di background in modo che sia capace di riconoscere eventuali segnali di nuova fisica come delle anomalie.\\


%Il Modello Standard è, senza ombra di dubbio, il fiore all'occhiello della fisica del Novecento. Tuttavia sono sempre più le evidenze che suggeriscono come esso si limiti a spiegare solo una parte della struttura profonda della natura: si è fatta strada sempre con più forza l'idea che esista una così detta fisica oltre il Modello Standard e, per indagarla, si costruiscono acceleratori di particelle sempre più potenti, di cui il Large Hadron Collider è l'esempio principale. 
%Il run3 del Large Hadron Collider è previsto per Maggio 2021, tuttavia non vi è stato un miglioramento notevole da un punto di vista energetico. Allo stesso tempo si stima che la produzione di dati sarà fino a dieci volte maggiore rispetto al run precedente, quindi ci si chiede se sia possibile trattare in maniera innovativa questa enorme mole di dati per provare a trovare segnale di nuova fisica. Nello specifico la domanda è se le metodologie di machine learning possano giocare un ruolo centrale per analizzare i dati prodotti nel prossimo ciclo di funzionamento di LHC.\\
%Con il termine machine learning si intende una serie di metodologie di natura statistico-computazionale che permettono di estrarre informazione utile da enormi moli di dati, altrimenti difficilmente processabili dall'uomo.
%I dati, per la loro stessa natura, sono disomogenei e caotici, quindi risulta particolarmente complesso analizzarli per ottenerne dei risultati. Qui entra in gioco il machine learning, ovvero l'apprendimento automatico della "macchina", perché permette di trovare relazioni nascoste fra i dati autonomamente, ovvero senza la continua supervisione dell'essere umano. Uno dei concetti fondamentali del machine learning è quello di apprendimento, che consiste nella possibilità di addestrare il modello in maniera iterativa. \\

\newpage


	


