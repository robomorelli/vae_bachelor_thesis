%
\section{Introduzione}\label{sec:introduzione}
%
Il Modello Standard è, senza ombra di dubbio, il fiore all'occhiello della fisica del Novecento. Tuttavia sono sempre più le evidenze che ci suggeriscono come il esso si limiti a spiegare solo una parte della struttura profonda della natura: si è fatta strada sempre con più forza l'idea che esista una così detta fisica oltre il Modello Standard e, per indagarla, si costruiscono acceleratori di particelle sempre più potenti, di cui il Large Hidron Collider è l'esempio principale. 
Il run3 del Large Hidron Collider è previsto per Maggio 2021, tuttavia non vi è stato un miglioramento notevole da un punto di vista energetico. Allo stesso tempo si stima che la produzione di dati sarà fino a dieci volte maggiore rispetto al run precedente, quindi ci si chiede se sia possibile trattare in maniera innovativa questa enorme mole di dati per provare a trovare segnale di nuova fisica. Nello specifico la domanda è se le metodologie di machine learning possano giocare un ruolo centrale per analizzare i dati prodotti nel prossimo ciclo di funzionamento di LHC.\\
Con il termine machine learning si intende una serie di metodologie di natura statistico-computazionale che permettono di estrarre informazione utile da enormi moli di dati, altrimenti difficilmente processabili dall'uomo.
I dati, per la loro stessa natura, sono disomogenei e caotici, quindi risulta particolarmente complesso analizzarli per ottenerne dei risultati. Qui entra in gioco il machine learning, ovvero l'apprendimento automatico della "macchina", perché permette di trovare relazioni nascoste fra i dati autonomamente, ovvero senza la continua supervisione dell'essere umano. Uno dei concetti fondamentali del machine learning è quello di apprendimento, che consiste nella possibilità di addestrare il modello in maniera iterativa.

%
