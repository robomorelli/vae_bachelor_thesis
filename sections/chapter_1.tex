%
\section{Apprendimento supervisionato}
\label{sec:apprendimento supervisionato}
In questa sezione viene portata avanti una descrizione più approfondita e formale dell'apprendimento supervisionato.\\
Come già accennato precedentemente, quando si parla di apprendimento supervisionato si hanno a disposizione sia gli input \textbf{x} che i corrispettivi target di output \textbf{y}; esisterà quindi una funzione 
\textbf{y} = f(\textbf{x}) che mette in relazione gli input con gli output. Tuttavia, come detto, tale funzione è incognita ed è quindi ciò che viene ricercato con l'algoritmo di apprendimento.\\
Nella pratica si cerca di approssimare la funzione agendo su una serie di parametri $\bm{\mu}$, quindi si avrà un qualcosa del tipo: \textbf{y'} = f'(\textbf{x},$\bm{\mu}$).


%
